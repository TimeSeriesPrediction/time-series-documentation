%%%%%%%%%%%%%%%%%%%%%%%%%%%%%%%%%%%%%%%%%%%%%%%%%%%%%%%%%%%%%%%%%%%%%%%%%%%%%%%%
%                              Document Settings                               %
%%%%%%%%%%%%%%%%%%%%%%%%%%%%%%%%%%%%%%%%%%%%%%%%%%%%%%%%%%%%%%%%%%%%%%%%%%%%%%%%

\documentclass[a4paper,12pt]{article}
\usepackage[utf8]{inputenc}
%%%%%%%%%%%%%%%%%%%%%%%%%%%%%%%%%%%%%%%%%%%%%%%%%%%%%%%%%%%%%%%%%%%%%%%%%%%%%%%%
%                                   Margins                                    %
%%%%%%%%%%%%%%%%%%%%%%%%%%%%%%%%%%%%%%%%%%%%%%%%%%%%%%%%%%%%%%%%%%%%%%%%%%%%%%%%

\addtolength{\oddsidemargin}{-1.cm}
\addtolength{\textwidth}{2cm}
\addtolength{\topmargin}{-2cm}
\addtolength{\textheight}{3.5cm}

%%%%%%%%%%%%%%%%%%%%%%%%%%%%%%%%%%%%%%%%%%%%%%%%%%%%%%%%%%%%%%%%%%%%%%%%%%%%%%%%
%                                   Packages                                   %
%%%%%%%%%%%%%%%%%%%%%%%%%%%%%%%%%%%%%%%%%%%%%%%%%%%%%%%%%%%%%%%%%%%%%%%%%%%%%%%%

\usepackage[pdftex]{graphicx}	% Include graphics
\usepackage{float}				% Place floats inline, use the [H] placement
\usepackage{array}				% Additional tables features


%%%%%%%%%%%%%%%%%%%%%%%%%%%%%%%%%%%%%%%%%%%%%%%%%%%%%%%%%%%%%%%%%%%%%%%%%%%%%%%%
%                                   Document                                   %
%%%%%%%%%%%%%%%%%%%%%%%%%%%%%%%%%%%%%%%%%%%%%%%%%%%%%%%%%%%%%%%%%%%%%%%%%%%%%%%%

\begin{document}
	
	% Title page
	% Title page
	\begin{titlepage}
		\begin{center}
			
			% University Logo
			\includegraphics[width=0.6\textwidth]{../images/up_logo.jpg}\\[2.0cm] 
			
			
			\textsc{\LARGE CS@UP Time Series Prediction}\\[1.0cm]
			
			
			\textsc{\Large System Requirements Specification Document}\\[0.75cm]
			
			
			\textsc{\Large COS301 - Capstone Project}\\[0.75cm]
			
			
			% Students/Contributors Table
			\textbf{\huge \\ Team:}
			\huge NewGen Leaders \\
			\begin{flushright} \large
				Claudio Da Silva		\emph{u14205892} \newline
				Dedr\'e Olwage	    	\emph{u15015239} \newline
				Merrisa Joubert			\emph{u15062440} \newline
				Murray Le Roux	    	\emph{u15311644} \newline
			\end{flushright}
			\small Department of Computer Science, University of Pretoria \\ 
			
			
		\end{center}
		
		% Report Declaration		
		\noindent By submitting this document we confirm that we have read and are aware of the University of Pretoria's policy on academic dishonesty and plagiarism and we declare that the work submitted in this assignment is our own as delimited by the mentioned policies. We explicitly declare that no parts of this assignment have been copied from current or previous students' work or any other sources (including the internet), whether copyrighted or not. We understand that we will be subjected to disciplinary actions should it be found that the work we submit here does not comply with the said policies.
		
		\begin{center}
			
			% Fill page
			\vfill
			
			% Date
			{\large \today}
			
		\end{center}
		
	\end{titlepage}
    
    \tableofcontents
    % An overview of the SRS document
    \section{Introduction}
    	
        % Specify the purpose of this SRS and the intended audience
        \subsection{Purpose}
        
        % Identify product by name, explain what the product will and will not do, describe the uses of the product including objectives, goals and benefits
The purpose of this document is to present a detailed description of the Time Series Prediction System. It will explain the purpose and features of the system, the interfaces of the system, what the system will do, the constraints under which it must operate and how the system will react to external stimuli. This document is intended for both the stakeholders and the developers of the system.
        \subsection{Scope}
The system will be used to enter, view, and edit student marks. It will include a prediction system that will use the student marks to make predictions on the outcomes of the students results at the end of the semester. These predictions will be used by lecturers to identify students who need extra help to pass the semester. The predictions may also be used to identify whether the teaching strategies of the lecturer are working as well as they expected it to. The system will allow students to view their marks and it will include an element of gamification that aims to improve student motivation. The ultimate goal of the system is to improve student results. The system also aims to improve the manner in which teaching assistants, tutors and lecturers record marks.
        \subsection{Definitions,Acronyms and Abbreviations}
        \begin{center}
\begin{tabular}{ |c|c| } 
\hline
Term & Definition \\
\hline
Stakeholder & A person, whom is not a developer, that has an interest in the product\\
\hline
Time series prediction & A number of algorithms that use current input data to predict future data\\
\hline
\end{tabular}
\end{center}
        \subsection{References}
        % Outline the rest of the SRS and how it is organized 
        This document defines the overall Description of the product. It then explains the product perspective and gives details about the different interfaces of the system. The document further explains the functions and characteristics of the system. System constraints and requirements are elaborated.
        \subsection{Overview}
        
    % Provide an overview of the product
    
    \section{Overall Description}
    
    	% Describe the context of the product and its relations and interfaces to other components of the total system. Block diagrams may be used to show the context and relationships. Describe also the characteristics and limits on primary and secondary memory, modes of operation, backup and recovery, and site specific requirements.
    	\subsection{Product perspective}
    	
        	\subsubsection{System Interfaces}
        	
            \subsubsection{User Interfaces}
            
            \subsubsection{Hardware Interfaces}
            
            \subsubsection{Software Interfaces}
            
            \subsubsection{Communication Interface}
            
            \subsubsection{Memory}
            
            \subsubsection{Operations}
            
            \subsection{Site Adaptation Requirements}
            
        \subsection{Product Functions}
        
        \subsection{User Characteristics}
        
        % Describe the restrictions on the solution space or options of the developer
        \subsection{Constraints}
               \begin{center}
        	\begin{tabular}{ |c|c| } 
        		\hline
        		Constraint Number & Constraint Description \\
        		\hline
        		C1 & Marks spreadsheets must be in a specific format \\
        		\hline
        		C2 & Only links to assignment pdfs can only be uploaded  \\
        		\hline
        		
        	\end{tabular}
        \end{center}
        %List the factors that affect the requirements
        \subsection{Assumptions and Dependencies}
        A number of factors that may affect the requirements specified in the SRS include:
        \begin{itemize}
        	\item It is assumed that there is an existing standard for marksheets and that all staff will follow this standard to ensure efficient processing of the marks.
        	\item It is assumed that there is a hierarchy structure of staff in place.
        	\item It is assumed that users will be part of the university system, and in order to use Saffron they must be a part of this system.
        	\item It is assumed that users will have internet access at all times when trying to use this system.
        	\item Students are assumed to know the processes of the university and all processes that apply to academics like marking and querying will apply to Saffron.
        	
        \end{itemize}
    \section{Specific Requirements}
    
    	\subsection{External Interface Requirements}
    	
        \subsection{Functional Requirements}
        	\subsubsection{Business Requirements}
        	\begin{itemize}
        	\item The system is initialized with a single Dean as the main Admin with all rights
        	\item The Dean adds HODs to the system
        	\item HODS can then add lecturers, and assistant lecturers to the system
        	\item Lecturers add teaching assistant, tutors, and students to their specific modules
        	\item Students write tests and complete practicals and assignments
        	\item The results of the tests, practicals and assignments are entered into the system by tutors, assistant lecturers and lecturers
        	\item The system performs time series prediction on these results and sends reports about the results and predictions to the lecturer
        	\item The lecturer may then choose to contact students who need extra help
        	\end{itemize}
        	\subsubsection{Security Requirements}
        	\begin{itemize}
        	\item The system will limit access to authorized users
        	\item Roles are assigned to user. Each role will have its own abilities. Roles include: student, teaching assistant, tutor, assistant lecturer, lecturer, head of department, and Dean.
        	\item Students can only view and query marks
        	\item Teaching assistants may add marks, but these marks have to be reviewed by a lecturer or assistant lecturer to be submitted to the database
        	\item Assistant Lecturers and lecturers may modify, view, and add or remove marks
        	\item The Dean has to add HODs to the system, HODS will add Lecturers, Lecturers will add assistant lecturers and assistant lecturers will add teaching assistants and tutors to the system
        	\end{itemize}
        	
        \subsection{Performance Requirements}
        \begin{itemize}
        \item The system will be required to work with potentially thousands of students.
        \item The system will manage many simultaneous notifications
        \item The system will need to manage the simultaneous login of a multitude of users.
        \end{itemize}
        
        \subsection{Design Constraints}
        \begin{itemize}
        \item Time limit
        \item The application will be web based and must be cross-platform
        \end{itemize}
        
        \subsection{Software System Attributes}
        	\subsubsection{Accessibility}
        	
        	\subsubsection{Accuracy}
        	\subsubsection{Administrability}
        	\subsubsection{Compatibility}
        	\subsubsection{Integrity}
        	\subsubsection{Maintainability}
        	\subsubsection{Mobility}
        	\subsubsection{Traceability}
        	\subsubsection{Vulnerability}
        	\subsubsection{Usability}
        \subsection{Other Requirements}
        \begin{itemize}
        \item Gamification elements are required to improve student motivation
        \item Anonymous leader boards will be displayed to all students
        \end{itemize}
    \section{Appendixes}
    
    \section{Index}
    
    \pagebreak  

\end{document}
