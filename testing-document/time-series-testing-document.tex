%%%%%%%%%%%%%%%%%%%%%%%%%%%%%%%%%%%%%%%%%%%%%%%%%%%%%%%%%%%%%%%%%%%%%%%%%%%%%%%%
%                              Document Settings                               %
%%%%%%%%%%%%%%%%%%%%%%%%%%%%%%%%%%%%%%%%%%%%%%%%%%%%%%%%%%%%%%%%%%%%%%%%%%%%%%%%

\documentclass[a4paper,12pt]{article}
\usepackage[utf8]{inputenc}
%%%%%%%%%%%%%%%%%%%%%%%%%%%%%%%%%%%%%%%%%%%%%%%%%%%%%%%%%%%%%%%%%%%%%%%%%%%%%%%%
%                                   Margins                                    %
%%%%%%%%%%%%%%%%%%%%%%%%%%%%%%%%%%%%%%%%%%%%%%%%%%%%%%%%%%%%%%%%%%%%%%%%%%%%%%%%

\addtolength{\oddsidemargin}{-1.cm}
\addtolength{\textwidth}{2cm}
\addtolength{\topmargin}{-2cm}
\addtolength{\textheight}{3.5cm}

%%%%%%%%%%%%%%%%%%%%%%%%%%%%%%%%%%%%%%%%%%%%%%%%%%%%%%%%%%%%%%%%%%%%%%%%%%%%%%%%
%                                   Packages                                   %
%%%%%%%%%%%%%%%%%%%%%%%%%%%%%%%%%%%%%%%%%%%%%%%%%%%%%%%%%%%%%%%%%%%%%%%%%%%%%%%%

\usepackage[pdftex]{graphicx}	% Include graphics
\usepackage{float}				% Place floats inline, use the [H] placement
\usepackage{array}				% Additional tables features


%%%%%%%%%%%%%%%%%%%%%%%%%%%%%%%%%%%%%%%%%%%%%%%%%%%%%%%%%%%%%%%%%%%%%%%%%%%%%%%%
%                                   Document                                   %
%%%%%%%%%%%%%%%%%%%%%%%%%%%%%%%%%%%%%%%%%%%%%%%%%%%%%%%%%%%%%%%%%%%%%%%%%%%%%%%%

\begin{document}
	
	% Title page
	% Title page
	\begin{titlepage}
		\begin{center}
			
			% University Logo
			\includegraphics[width=0.6\textwidth]{../images/up_logo.jpg}\\[2.0cm] 
			
			
			\textsc{\LARGE CS@UP Time Series Prediction}\\[1.0cm]
			
			
			\textsc{\Large Testing Document}\\[0.75cm]
			
			
			\textsc{\Large COS301 - Capstone Project}\\[0.75cm]
			
			
			% Students/Contributors Table
			\textbf{\huge \\ Team:}
			\huge NewGen Leaders \\
			\begin{flushright} \large
				Claudio Da Silva		\emph{u14205892} \newline
				Dedr\'e Olwage	    	\emph{u15015239} \newline
				Merrisa Joubert			\emph{u15062440} \newline
				Murray Le Roux	    	\emph{u15311644} \newline
			\end{flushright}
			\small Department of Computer Science, University of Pretoria \\ 
			
			
		\end{center}
		
		% Report Declaration		
		\noindent By submitting this document we confirm that we have read and are aware of the University of Pretoria's policy on academic dishonesty and plagiarism and we declare that the work submitted in this assignment is our own as delimited by the mentioned policies. We explicitly declare that no parts of this assignment have been copied from current or previous students' work or any other sources (including the internet), whether copyrighted or not. We understand that we will be subjected to disciplinary actions should it be found that the work we submit here does not comply with the said policies.
		
		\begin{center}
			
			% Fill page
			\vfill
			
			% Date
			{\large \today}
			
		\end{center}
		
	\end{titlepage}
    
    \tableofcontents
    
    \pagebreak
    
    % An overview of the Testing
    \section{Introduction}
    	
        %Specify the purpose of this SRS and the intended audience
        \subsection{Purpose}
        
        The purpose of this document is to show the level to which the system has been tested, as well as document problem areas and other concerns that may arise from testing. In the event that something in future may break, this document will be the proof that at the current time, the tests were indeed working.
        
        %Identify product by name, explain what the product will and will not do, describe the uses of the product including objectives, goals and benefits
        \subsection{Scope}
        
       The system will be used to enter, view, and edit student marks. It will include a prediction system that will use the student marks to make predictions on the outcomes of the students results at the end of the semester. These predictions will be used by lecturers to identify students who need extra help to pass the semester. The predictions may also be used to identify whether the teaching strategies of the lecturer are working as well as they expected it to. The system will allow students to view their marks and it will include an element of gamification that aims to improve student motivation. The ultimate goal of the system is to improve student results. The system also aims to improve the manner in which teaching assistants, tutors and lecturers record marks.
        
        \subsection{Definitions, Acronyms and Abbreviations}
        
       \begin{tabular}{ |c|c| } 
        	\hline
        	Term & Definition \\
        	\hline
        	NPM & The Node Package Manager, used to interact with node packages and scripts\\
        	\hline
        	CLI & A Command Line Interface, used to manipulate a program through command line\\
        	\hline
        	NYC & The CLI for the Istanbul code coverage module\\
        	\hline
        \end{tabular}
    
    \pagebreak
    
    % Explain the current testing framework in use
    \section{Testing framework}
    
    	% Describe the testing framework in place for the server
    	\subsection{Server}
    	
    		% Describe the chosen testign framework in place
	    	\subsubsection{The chosen testing framework}
	    	
	    	For server side, we make use of the jasmine testing framework for nodeJS. We also use Istanbul in order to check for test coverage, via their CLI, NYC.
	    	
	    	% Explain how to run the tests for the that framework
	        \subsubsection{How to run the tests}
	        
	        To run the tests, simply go to the root folder of the application, open up a command line instance and type, "npm test".
	        
	        % Explain the current system of automated testing and how to use and view it
	        \subsubsection{Automated testing overview}
	        
	        Tests are run on a GitHub push basis using the Travis Continuous Integration tool. A weekly check is also run to ensure everything stays in order. Code Climate is then used to determine the quality of code being written, as well as document the level of code that is covered, as determined by Istanbul.
	        
	    	% Describe the testing framework in place for the client
	    \subsection{Web Client}
	    
	    % Describe the chosen testign framework in place
	    \subsubsection{The chosen testing framework}
	    
    	For client side, we make use of the jasmine testing framework for nodeJS. We however make use of Karma to run jasmine, as a more Angular friendly approach. Protractor is also used to run tests that simulate user input.
	    
	    % Explain how to run the tests for the that framework
	    \subsubsection{How to run the tests}
	    
	     To run the tests, simply go to the root folder of the application, open up a command line instance and type, "npm test" or alternatively "ng test".
	    
	    % Explain the current system of automated testing and how to use and view it
	    \subsubsection{Automated testing overview} 
	    
	    Tests are run on a GitHub push basis using the Travis Continuous Integration tool. A weekly check is also run to ensure everything stays in order.
      
    \pagebreak 
    % List the tests performed on the system
    \section{System tests}
    
    	% List the unit tests which test the functions in the system
    	\subsection{Unit tests}
    	
    		\subsubsection{Server}
    		
   			\begin{tabular}{ |c|c|c| } 
    			\hline
    			Test number & Test overview & Test result \\
    			\hline
    			1 & Forgot password sends right URL & Passing\\
    			\hline
    			2 & GetUsers returns a list of users & Passing\\
    			\hline
    			3 & Validation fails if username is empty & Passing\\
    			\hline
    			4 & Validation fails if password is empty & Passing\\
    			\hline
    			5 & Validation fails if email is empty & Passing\\
    			\hline
    			6 & Correct password validated & Passing\\
    			\hline
    			7 & Incorrect password invalidated & Passing\\
    			\hline
    			8 & Invalidate incorrect password casing & Passing\\
    			\hline
    			9 & Invalidate extra spaces in password & Passing\\
    			\hline
    			10 & Authenticates if token valid & Passing\\
    			\hline
    			11 & Sets the user in the request object & Passing\\
    			\hline
    			12 & Unauthorized access if JSON invalid & Passing\\
    			\hline
    			13 & Unauthorized access if not encrypted & Passing\\
    			\hline
    			14 & Unauthorized access if token expired & Passing\\
    			\hline
    			15 & Unauthorized access if IP does not match & Passing\\
    			\hline
    			16 & Unauthorized access if token not set & Passing\\
    			\hline
    			17 & Generate token should return encrypted token & Passing\\
    			\hline
    			18 & Encrypt/decrypt should work on data & Passing\\
    			\hline
    			19 & GetSalt should return same length salt & Passing\\
    			\hline
    			20 & GetSalt should append salt to hash & Passing\\
    			\hline
    			21 & GetSalt should compute same hash with same data & Passing\\
    			\hline
    			22 & GenerateResetToken should be URL safe & Passing\\
    			\hline
    			23 & GenerateResetToken should be 20 characters long & Passing\\
    			\hline
    		\end{tabular}
    		
    		\subsubsection{Web Client}
    		
   			\begin{tabular}{ |c|c|c| } 
    			\hline
    			Test number & Test overview & Test result \\
    			\hline
    			1 & App Component was successfully created & Passing\\
    			\hline
    			2 & Authorization Guard was successfully created & Passing\\
    			\hline
    			3 & Authorization Service was successfully created & Passing\\
    			\hline
    		\end{tabular}
    	
    	% List the integration tests which test compatability between modules
        \subsection{Integration tests}
        
        	\subsubsection{Server}
        	
        	\subsubsection{Web Client}
        
        % List the functional/system test which test functionality between containers, and overall working of the system
        \subsection{Functional/system tests}
	
	\pagebreak
	\section{Testing release log}
	
	\begin{itemize}
		\item \textbf{Server}
		\begin{itemize}
			\item
				\begin{tabular}{ |c|c|c| } 
					\hline
					Date & Number of tests & Number of passing tests \\
					\hline
					2017/07/28 & 23 & 23\\
					\hline
				\end{tabular}
		\end{itemize}
 	\end{itemize}
 
	\begin{itemize}
 		\item \textbf{Client}
 		\begin{itemize}
 			\item
 			\begin{tabular}{ |c|c|c| } 
 				\hline
 				Date & Number of tests & Number of passing tests \\
 				\hline
 				2017/07/28 & 3 & 3\\
 				\hline
 			\end{tabular}
 		\end{itemize}
 	\end{itemize}  
   
    \section{Appendixes}
    
    \section{Index}
    \pagebreak  

\end{document}
